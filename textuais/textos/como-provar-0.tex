Considerando funções Booleanas do tipo $f: \Omega_n \longrightarrow \{0, 1\}$, é fundamental que sejamos capazes de estudar o comportamento de $\EX_p(f(\omega))$ para, de acordo com a definição de \textit{sharp threshold} apresentada na Seção \ref{Intoduction}, entender a transição de fase pela qual o sistema modelado passa. Dessa forma, ao longo dessa seção vamos estabelecer cotas para quantidades de interesse que envolvem $\frac{d}{dp}\EX_p(f(\omega))$.

%{\color{red} Lorem Ipsum is simply dummy text of the printing and typesetting industry. Lorem Ipsum has been the industry's standard dummy text ever since the 1500s, when an unknown printer took a galley of type and scrambled it to make a type specimen book. It has survived not only five centuries, but also the leap into electronic typesetting, remaining essentially unchanged.}

