\newpage

\phantomsection
\section*{\Large \hspace*{\fill} \textbf{Abstract} \hspace*{\fill}}
\addcontentsline{toc}{section}{Abstract}

The \textit{phase transition} concept, intrinsic to some purely deterministic models, may also be seen in systems with some stochastic component. Therefore, it is important to study results that allow us to prove that some arbitrary random systems present such a characteristic.

In this regard, given an appropriate probability space, results associated with Boolean function analysis play an important role in the study of this class of models. As a consequence of it, this text focuses on, throughout Section \ref{How-to-prove}, introduce and prove such results. 

With respect to independent random models from Statistical Physics, the Bernoulli Percolation Model may be considered the most popular one. Thus, in Section \ref{Bernoulli-percolation}, we focused on replicating some ``classical'' results concerned with it. In order to achieve this, we used the tools developed in Section \ref{How-to-prove}. At this point, it is important to stress the benefits of adopting such an approach. Some of the proofs may be extended, through similar strategies, to models defined over more general spaces $-$ which also includes models with a structure of dependence, as discussed in Section \ref{Monotonic-measures}.

Finally, I would like to clarify that the results presented throughout this text are \textbf{not} original. This work was developed mainly based on \cite{duminil2019sharp} $-$ in addition to other resources and academic articles, which were properly cited.

\vspace{12pt}
\par Keywords: phase transition; sharp threshold; Boolean functions analysis; percolation.
