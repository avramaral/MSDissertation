\newpage

\phantomsection
\section*{\Large \hspace*{\fill} \textbf{Resumo} \hspace*{\fill}}
\addcontentsline{toc}{section}{Resumo}

O conceito de \textit{transição de fase}, inerente a diferentes tipos de modelos puramente determinísticos, pode, também, ser verificado em sistemas com componentes estocásticas. Nesse sentido, é importante o estudo de ferramentas que nos permitem provar que determinados sistemas aleatórios apresentam esse tipo de característica. 

Assim, dado um espaço de probabilidade apropriado, resultados associados à análise de funções Booleanas desempenham papel importante no estudo dessa classe de modelos. Este texto preocupa-se em, ao longo da Seção \ref{How-to-prove}, apresentar e demonstrar tais resultados.

Tratando-se de modelos aleatórios independentes e que vêm da Física Estatística, o de Percolação Bernoulli é, talvez, o mais conhecido. Por isso, na Seção \ref{Bernoulli-percolation}, nos concentramos em reproduzir alguns resultados ``clássicos'' desse modelo; utilizando, porém, as ferramentas desenvolvidas na Seção \ref{How-to-prove}. Aqui, é importante ressaltar o ganho que existe em utilizar esse tipo de abordagem. Alguns dos resultados demonstrados poderão ser estendidos, utilizando-se de estratégias similares, para modelos construídos sobre espaços mais gerais $-$ incluindo modelos com dependência, como os que foram discutidos na Seção \ref{Monotonic-measures}.

Por fim, deixo o registro de que os resultados apresentados ao longo desse texto \textbf{não} são originais. O trabalho foi construído a partir de, principalmente, \cite{duminil2019sharp} $-$ em adição aos outros artigos e recursos que foram apropriadamente referenciados.

\vspace{12pt}

\par Palavras-chave: transição de fase; \textit{sharp threshold}; análise de funções Booleanas; percolação.
